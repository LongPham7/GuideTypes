% !TeX root = README.tex

\section{Getting Started Guide}

The artifact is wrapped inside the accompanying Docker image
\texttt{guide\_types.tar.gz}.
%
Before running it, first install Docker as instructed here:
\url{https://docs.docker.com/engine/install/}.
%
To see if Docker has been installed properly, run
\begin{verbatim}
$ docker --version
Docker version 27.0.3, build 7d4bcd8
\end{verbatim}

Load a Docker image by running
\begin{verbatim}
$ docker load --input guide_types.tar.gz
\end{verbatim}
%
It creates an image named \texttt{guide\_types} and stores it locally on your
%
Docker may create an image with a slightly different name from
\texttt{guide\_types}.
%
To check the name of the image, display all Docker images on your local machine
by running
\begin{verbatim}
$ docker images
\end{verbatim}

To run the image \texttt{guide\_types}, run
\begin{verbatim}
$ docker run --name guide_types -it --rm guide_types
root@5b1c8c873064:/home/GuideTypes#
\end{verbatim}
%
It creates a Docker container (i.e., a runnable instance of the Docker image)
named \texttt{guide\_types} and starts a shell inside the container.
%
If the command does not run properly, you can instead build the image locally on
your machine as instructed in \cref{sec:Build the Docker Image}.

Throughout this document, any command line starting with \texttt{\#} is executed
inside the Docker container, and any command line starting with \texttt{\$} is
executed in your local machine's terminal.

\subsection{Structural-Type-Equality Checking}

The initial working directory \texttt{GuideTypes/bench/type-equality} of the
Docker container has a file \texttt{type-equality-sample}, which stores three
guide-type definitions.
%
Running the command
\begin{verbatim}
# cat type-equality-sample
\end{verbatim}
prints out the content of the file:
\begin{verbatim}
type T1 = &{ $ | real /\ T1[T1] }
type T2 = &{ $ | real /\ T2[T2] }
type T3 = &{ $ | real /\ real /\ T3[T3] }
\end{verbatim}
%
The first line defines a guide-type operator $T_1 [\cdot]$ as
\begin{equation*}
  T_1 [X] \coloneq X \echoice (\treal \land \treal \land T_1 [T_1[X]]),
\end{equation*}
where $X$ is a type variable that stands for a continuation type (i.e., the type
of the communication protocol that we run after $T_1$ is finished).
%
The type $T_1$ means the coroutine receives a branch selection (from another
coroutine that it communicates with), and proceeds to either $X$ (i.e., the
continuation guide type) or $\treal \land T_1 [T_1[X]]$, depending on which
branch is taken.
%
The guide type $\treal \land T_1 [T_1[X]]$ means the coroutine first sends a
message of type $\treal$ (i.e., real numbers) and then proceeds to the guide
type $T_1 [T_1 [X]]$.
%
Here, $T_1 [T_1 [X]]$ can be interpreted as a sequential composition of two
instances of $T_1$, followed by the original continuation $X$.

To algorithmically decide the structural type equality between the guide-type
operators $T_1 [X]$ and $T_2 [X]$, run
\begin{verbatim}
# dune exec gtypes type-equality type-equality-sample T1 T2
\end{verbatim}
%
The command first prints out information about intermediate results of the
type-equality checking algorithm for debugging.
%
At the end of the printout, the commmand displays the result of
structural-type-equality checking:
\begin{verbatim}
Types T1 and T2 are equal
\end{verbatim}
%
This result is as expected because the two type operators $T_1 [X]$ and $T_2
  [X]$ represent the same type---they only differ in the names of type operators.

Next, to check the structural type equality between $T_1 [X]$ and $T_3 [X]$, run
\begin{verbatim}
# dune exec gtypes type-equality type-equality-sample T1 T3
\end{verbatim}
%
At the end of the printout, it displays:
\begin{verbatim}
Types T1 and T3 are unequal
\end{verbatim}
%
This is a correct result because the right branch of $T_1 [X]$ only sends one
value (indicated by $\treal \land {}$) before possible termination, while the
right branch of $T_3 [X]$ sends two values before possible termination.

\subsection{Guide-Type Inference}
\label{sec:Guide-Type Inference}

Go to the directory
\texttt{/home/GuideTypes/bench/coverage-check/recursive/recur} by running
\begin{verbatim}
# cd /home/GuideTypes/bench/coverage-check/recursive/recur
\end{verbatim}
%
It contains a benchmark program \texttt{recur-covered-aligned}.
%
To display its content, run
\begin{verbatim}
# cat recur-covered-aligned
\end{verbatim}

The first three lines of the file are
\begin{verbatim}
type Old_trace
type General
type General_no_old
\end{verbatim}
%
They declares three type operators that are later mentioned in the type
signatures of coroutines.
%
While the user needs to manually declare these type operators and insert them to
the guide programs' type signatures, the type operators' definitions will be
automatically inferred by our artifact.

The file defines three coroutines: \texttt{Recur1}, \texttt{Recur2},
and \texttt{Recur\_no\_old}.
%
In each of them, the first line states the type signature of the coroutine.
%
For example, the type signature of the the coroutine \texttt{Recur1} is
\begin{verbatim}
proc Recur1() -> unit | old : Old_trace | lat : General
\end{verbatim}
%
It states (i) the coroutine \texttt{Recur1} has the output type of
\texttt{unit}, (ii) the coroutine uses a channel named \texttt{old} of the guide
type \texttt{Old\_trace}, and (iii) the coroutine uses a channel named
\texttt{lat} of the guide type \texttt{lat}.

To infer the definitions of the guide types mentioned in this file, we run
\begin{verbatim}
# dune exec gtypes type-check recur-covered-aligned
\end{verbatim}
%
It prints out the inferred definitions of the three type operators.
%
For instance, the inference result for the type operator \texttt{General} is
stated as:
\begin{verbatim}
successfully inferred guide type operator General[$]
	&{ $ | real /\ General[real /\ General[real /\ General[$]]] }
\end{verbatim}
%
It translates to
\begin{equation*}
  \texttt{General} [X] \coloneq X \echoice (\treal \land \texttt{General} [\treal \land \texttt{General} [\treal \land \texttt{General} [X]]]).
\end{equation*}

In addition to the inference results of guide types, the command also prints out
(i) a list of pairs of guide types whose structural type equality we need to
check during type inference and (ii) whether each pair is indeed equal.
%
An example is
\begin{verbatim}
Types $ and $ are equal modulo coverage
\end{verbatim}
%
At the end of the command's output, it displays \texttt{All equal modulo
  coverage}.
%
It means all type-equality conditions discharged during type inference are
verified, and hence the three inferred type definitions are also valid.

\subsection{Coverage Checking}

The last two liens in the benchmark program \texttt{recur-covered-aligned}
define how the guide coroutines are sequentially composed:
\begin{verbatim}
Initial_type: Init_type
Guide_composition: Recur1; Recur2
\end{verbatim}
%
The first line states the initial guide type \texttt{Init\_type}, which is
defined in the fourth line in the file:
\begin{verbatim}
type Init_type = &{ $ | real_u /\ Init_type[real_u /\
                        Init_type[real_u /\ Init_type[$]]] }
\end{verbatim}
%
This guide type describes the initial trace fed to the sequential composition of
guide coroutines.
%
Because we assume the initial trace is uncovered (i.e., all random variables in
the trace haven't been freshly sampled), the guide type \texttt{Init\_type}
contains \texttt{real\_u}, where the subscript \texttt{u} means ``uncovered.''

Finally, the last line in the file, \texttt{Guide\_composition: Recur1; Recur2},
means we sequentially compose the guide coroutines \texttt{Recur1} and
\texttt{Recur2} in this order.

To verify the full coverage of the sequential composition of the two guide
coroutines, we run
\begin{verbatim}
# dune exec gtypes coverage-check recur-covered-aligned
\end{verbatim}
%
The result is displayed at the end of the printout:
\begin{verbatim}
The final type is fully covered
\end{verbatim}
