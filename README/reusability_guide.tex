% !TeX root = README.tex

\section{Reusability Guide}

This section provides an overview of the artifact's code base and explains how
to build the Docker image.

\subsection{Source Code}
\label{sec:Source Code}

The OCaml source code of this artifact is stored in the project directory
\texttt{/home/GuideTypes} inside the Docker container.
%
Our artifact builds on the codebase of the prior work by
\citeauthor{PLDI:WHR21B}.
%
The structural-type-equality checking (\ref{functionality:type-equality
checking}) is implemented in the OCaml source file
\texttt{type\_equality\_check.ml}.
%
The coverage-checking algorithm (\ref{functionality: coverage checking}) is
implemented in the source file \texttt{coverage\_checking.ml}.
%
The source file \texttt{typecheck.ml} for type inference from the prior work's
artifact (\ref{functionality:type inference}) has been modified to incorporate
the structural-type-equality checking.

\subsection{Build the Docker Image}
\label{sec:Build the Docker Image}

The code for building a Docker image is available on GitHub:
\url{https://github.com/LongPham7/GuideTypes/tree/subguide_types}.
%
To build a Docker image, clone the GitHub repository and then run (in the root
directory)
\begin{verbatim}
$ docker build -t guide_types .
\end{verbatim}
%
We need a period at the end of the command to indciate that a
\texttt{Dockerfile} exists in the current working directory.
%
The build will take 10--20 minutes.
%
The resulting image is named \texttt{guide\_types} and is stored locally on your
machine.

To run the image \texttt{guide\_types}, run
\begin{verbatim}
$ docker run --name guide_types -it --rm guide_types
\end{verbatim}
%
It creates and runs a Docker container with the same name \texttt{guide\_types}.
%
If you want to save the image as a tar archive and compress it, run
\begin{verbatim}
$ docker save guide_types | gzip > guide_types.tar.gz
\end{verbatim}
