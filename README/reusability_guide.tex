% !TeX root = README.tex

\section{Reusability Guide}

This section provides an overview of the artifact's code base and explains how
to build the Docker image.

\subsection{Source Code}
\label{sec:Source Code}

The OCaml source code of this artifact is stored in the project directory
\texttt{/home/GuideTypes} inside the Docker container.
%
Our artifact builds on the codebase of the prior work\footnotemark by
\citeauthor{PLDI:WHR21B}.

The structural-type-equality checking (\labelcref{functionality:type-equality
  checking}) is implemented in the OCaml source file
\texttt{type\_equality\_check.ml}.
%
The coverage-checking algorithm (\labelcref{functionality: coverage checking})
is implemented in the source file \texttt{coverage\_checking.ml}.
%
The source file \texttt{typecheck.ml} for type inference from the prior work's
artifact (\labelcref{functionality:type inference}) has been modified to
incorporate the structural-type-equality checking.

\footnotetext{Available on GitHub:
  \url{https://github.com/stonebuddha/GuideTypes/tree/pldi21}.}

\subsection{Build the Docker Image}
\label{sec:Build the Docker Image}

The code for building a Docker image is available on GitHub:
\url{https://github.com/LongPham7/GuideTypes/tree/subguide_types}.
%
To build a Docker image, clone the GitHub repository and then run (in the root
directory)
\begin{verbatim}
$ docker build -t guide_types .
\end{verbatim}
%
We need a period at the end of the command to indciate that a
\texttt{Dockerfile} exists in the current working directory.
%
The build will take 10--20 minutes.
%
The resulting image is named \texttt{guide\_types} and is stored locally on your
machine.

To run the image \texttt{guide\_types}, run
\begin{verbatim}
$ docker run --name guide_types -it --rm guide_types
\end{verbatim}
%
It creates and runs a Docker container with the same name \texttt{guide\_types}.
%
If you want to save the image as a tar archive and compress it, run
\begin{verbatim}
$ docker save guide_types | gzip > guide_types.tar.gz
\end{verbatim}

\subsection{Language Syntax}

This section describes the syntax of our probabilistic programming language
(PPL) used to write input programs.
%
We adapt the syntax from the prior work's artifact~\citep{PLDI:WHR21B} by
extending with several additional constructs.

The language consists of two layers: a functional layer and a coroutine layer.
%
The functional layer is a standard functional programming language augmented
with probability distributions.
%
The user uses this layer to write \emph{expressions}.
%
The coroutine layer defines \emph{commands}, which describe how to communicate
with other coroutines by exchanging messages on channels.

\subsubsection{Functional Layer}

\paragraph{Functional types}

The functional layer offers the following primitive types:
\begin{itemize}
  \item \texttt{unit} for the unit type
  \item \texttt{bool} for Booleans
  \item \texttt{nat[n]} for natural numbers smaller than $n \in \bbN$
  \item \texttt{nat} for natural numbers
  \item \texttt{ureal} for real numbers on the unit interval
  \item \texttt{preal} for positive real numbers
  \item \texttt{real} for real numbers.
\end{itemize}
%
The functional layer also proivdes the following compound types:
\begin{itemize}
  \item \texttt{t1 -> t2} for an arrow type $\tau_1 \to \tau_2$
  \item \texttt{t1 * t2} for a product type $\tau_1 \times \tau_2$
  \item \texttt{t dist} for a distribution where $\tau$ is the type of the
        support of the distribution
  \item \texttt{(t, [m1; ...; mn]) tensor} for a tensor of an element type
        $\tau$ and a shape $[m_1, \ldots, m_n]$.
\end{itemize}

\paragraph{Expressions}

The supported probability distributions are
\begin{itemize}
  \item \texttt{BER(p)}: Bernoulli distribution with a parameter $p \in [0, 1]$
  \item \texttt{UNIF}: the uniform distribution on the unit interval
  \item \texttt{BETA(alpha; beta)}: beta distribution with shape parameters
        $\alpha, \beta > 0$
  \item \texttt{GAMMA(alpha; beta)}: gamma distribution with a shape parameter
        $\alpha > 0$ and a rate $\beta > 0$
  \item \texttt{Normal(mu; sigma)}: normal distribution with a location
        parameter $\mu$ and a scale parameter $\sigma >0$
  \item \texttt{CAT(p1; ...; pn)}: categorical distribution with probabilities
        $p_1, \ldots, p_n$
  \item \texttt{BIN(n; p)}: binomial distribution with parameters $n \in \bbN$
        and $p \in [0, 1]$
  \item \texttt{GEO(p)}: geometric distribution with a parameter $p \in [0, 1]$
  \item \texttt{POIS(r)}: Poisson distribution with a rate $r \in (0, \infty)$.
\end{itemize}

The functional layer supports the following expressions:
\begin{itemize}
  \item Binary operators, including arithmetic operators (e.g., \texttt{+} for
        addition), comparison operators (e.g., $<=$ for $\leq$), and logical
        operators (e.g., \texttt{and} for logical conjunction)
  \item Function abstraction \texttt{fn (x: t) -> e}, where the input variable
        $x$ has type $\tau$ and the function body is $e$
  \item Function application \texttt{e1 (e2)}, where a function $e_1$ is applied
        to an expression $e_2$
  \item Conditional expression \texttt{if e then e1 else e2}.
\end{itemize}

\subsubsection{Coroutine Layer}

In the coroutine layer, commands are used to send and receive random samples
between coroutines across channels.
%
Guide types describe communication protocols on channels.

\paragraph{Guide types}

The language provides the following guide types:
\begin{itemize}
  \item \texttt{T} for a unary type operator $T$
  \item \texttt{\$} for a type variable.
        %
        We do not need to differentiate between different type variables (e.g.,
        $X$ and $Y$), because we only support univariate type operators (i.e.,
        each type operator comes only with one type variable).
  \item \verb$t /\ A$ for a guide type $\tau \land A$.
        %
        It sends a value of a functional type $\tau$ and proceeds to a guide
        type $A$.
  \item \texttt{t -o A} for a guide type $\tau \supset A$.
        %
        It receives a value of a functional type $\tau$ and proceeds to a guide
        type $A$
  \item \verb$+{A1 | A2}$ for an internal choice $A_1 \oplus A_2$.
        %
        It chooses between $A_1$ and $A_2$ and proceeds with it.
  \item \verb$&{A1 | A2}$ for an external choice $A_1 \mathrel{\&} A_2$.
        %
        It receives a branch selection and then proceeds to either $A_1$ or
        $A_2$ depending on the branch selection.
  \item \texttt{T[A]} for an instantiation of a type operator $T$ with a
        continuation guide type $A$.
\end{itemize}
%
In the guide types $t \land A$ and $t \supset A$, the type $\tau$ of the value
being transferred is either covered (i.e., the value is freshly sampled from
some probability distribution) or uncovered (i.e., an old value is reused for
this random variable).
%
Covered types have same notation as the primitive types (e.g., \texttt{nat[n]}
and \texttt{real}).
%
Uncovered types, on the other hand, are denoted by primitive types with a
subscript \texttt{u} (e.g., \texttt{nat\_u[n]} and \texttt{real\_u}).

\paragraph{Commands}

The language supports the following commands:
\begin{itemize}
  \item \texttt{return e} evaluates an expression $e$ and returns its value as
        output.
  \item \texttt{x <- C1; C2} executes a command $C_1$, binds its output value to
        variable $x$, and proceeds to a command $C_2$.
  \item \texttt{Pname(e1; ...; en)} calls a procedure \texttt{Pname} with
        arguments $e_1, \ldots, e_n$.
  \item \verb$sample{ch}(e)$ evaluates an expression $e$ to get a distribution
        $d$, draws a random sample $v$ from the distribution $d$, and sends $v$
        across a channel \texttt{ch}.
  \item \verb$sample{ch}{SAME(t)}$ reuses an old value (of type $\tau$) from the
        old trace for this random variable.
        %
        This command is only applicable to guide coroutines.
        %
        Model coroutines do not use it.
  \item \verb$observe{ch}(e)$ evaluates an expression $e$ to get a distribution
        $d$, receives a value $v$ across a channel \texttt{ch}, and proceeds as
        if $v$ is sampled from $d$.
  \item \verb$if{ch} e then C1 else C2$ evaluates an expression $e$ to get a
        Boolean $b$, picks a command $C_1$ or $C_2$ based on $b$, and sends $b$
        across a channel \texttt{ch}.
  \item \verb$if{ch} . then C1 else C2$ receives a Boolean $b$ from a channel
        \texttt{ch}, and proceeds to a command $C_1$ or $C_2$ based on the value
        of $b$.
  \item \texttt{if e then C1 else C2} evaluates an expression $e$ to a Boolean
        $b$, and proceeds to a command $C_1$ or $C_2$ depending on $b$.
  \item \texttt{loop[n; einit](fn (x : t) -> C)} loops for $n$ iterations with
        an accumulator of functional type $\tau$ and initial value
        \texttt{einit}.
        %
        In each iteration, it binds the accumulator from the last iteration to a
        variable $x$ and executes a command $C$, whose return value becomes the
        next accumulator.
  \item \texttt{iter[e; einit](fn t (x : t) -> C)} evaluates $e$ to a tensor,
        and then loops through the tensor along its first dimension with an
        accumulator of functional type $\tau$ and initial value \texttt{einit}.
        %
        In each iteration, it binds the sub-tensor to $t$ and the accumulator
        from the last iteration to a variable $x$, then executes a command $C$.
\end{itemize}

\subsubsection{Sequential Composition of Guide Coroutines}

An input file to the artifact has three components:
\begin{itemize}
  \item A list of type declarations (i.e., we simply declare the names of type
        operations without defining them) and type definitions (i.e., we declare type
        operations and also provide their definitions).
        %
        For a type declaration of a type operator \texttt{T}, we write \texttt{type T}
        and for a type definition, we write \texttt{type T = <type-definition>}.
  \item A list of procedure definitions, each of which contains a type signature
        (i.e., output type, channel names, and their guide-type operators) and the
        procedure body.
  \item An initial (uncovered) guide type.
        %
        The syntax is
        \begin{verbatim}
Initial_type: <initial-guide-type>
\end{verbatim}
        %
        It describes the initial trace fed to a sequential composition of guide
        coroutines.
  \item A sequential composition of procedures for guide coroutines.
        %
        The syntax is
        \begin{verbatim}
Guide_composition: Guide1; Guide2; ...; Guiden
\end{verbatim}
\end{itemize}
