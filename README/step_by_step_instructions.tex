% !TeX root = README.tex

\section{Step-by-Step Instructions}

\subsection{Benchmark Suite}

The artifact contains twenty-eight benchmark programs as listed in Table~1 of
the paper.
%
To construct these programs, we have modified the benchmark programs from the
prior work~\citep{PLDI:WHR21B} by sequentially composing multiple guide
coroutines, each of which updates a different subset/block of random variables.
%
Each benchmark has its own subdirectory inside the directory
\texttt{GuideTypes/bench/coverage-check}.

A benchmark's subdirectory contains two files: one for the covered version and
another for the uncovered version.
%
In the covered version, the random variables are freshly sampled in such a way
that every random variable is sampled by at least one guide coroutine in all
execution paths; that is, all random variables are ``covered.''
%
Meanwhile, in the uncovered version, in some execution path, not all random
variables are freshly sampled.
%
Both versions have the same code structure---they only differ in whether, for
each individual random variable, we freshly sample it or reuse an old value from
the previous trace.

Additionally, for the four context-free benchmark programs listed at the bottom
of Table~1 (i.e., \texttt{ex-2}, \texttt{diter}, \texttt{gp-dsl}, and
\texttt{recur}), they have aligned and misaligned versions of the source code
(for each of the covered and uncovered versions).
%
In the aligned version, all guide coroutines in a sequential composition have
the same code structure with respect to procedure calls.
%
This assumption is necessary for the coverage-checking algorithm to handle guide
coroutines with context-free guide types.
%
If the violation of this assumption is detected during coverage checking, the
algorithm raises an exception, which we interpret as Uncovered/Negative.
%
This can result in \textcolor{red}{False Negatives}, which are color-highlighted
in red in Table~1.

\subsection{Guide-Type Inference}

To produce the analysis time of guide-type inference (i.e., the fifth column of
Table~1), in the directory \texttt{GuideTypes/bench/coverage-check}, run
\begin{verbatim}
# python3 type_inference_table_result.py
\end{verbatim}
%
It runs the command \texttt{dune exec gtypes type-check} (\cref{sec:Guide-Type
  Inference}) for the covered version of each benchmark program.
%
We do not need to infer guide types for the uncovered version of the programs,
because it has the same code structure as the covered version.

The command prints out a table of analysis time:
\begin{verbatim}
Name                       Path                                   LOC  Check Time
-------------------------  -----------------------------------  -----  ------------
branching                  branching/branching-covered             46  3.926ms
...
recur-covered-misaligned   recur/recur-covered-misaligned          83  29.103ms
\end{verbatim}
%
In the table, the unit \texttt{us} means microsecond, and the unit \texttt{ms}
means millisecond.
%
Although the analysis time reported in this table may be different from the one
reported in Table~1 of the paper, the difference should not be significant.

Additionally, the command writes out the type-inference results a log file
\begin{verbatim}
GuideTypes/bench/coverage-check/table-type-inference.log
\end{verbatim}
%
To view its content, run
\begin{verbatim}
# cat table-type-inference.log
\end{verbatim}
% To transfer it from the Docker container to your local machine, run the
% following command on the local machine's terminal:
% \begin{verbatim}
% $ docker cp guide_types:/home/GuideTypes/bench/coverage-check/table-type-inference.log\
% <path-to-local-directory>
% \end{verbatim}

\subsection{Coverage Checking}

To produce the results and analysis time of coverage checking (i.e., the last
three columns of Table~1), in the directory
\texttt{GuideTypes/bench/coverage-check}, run
\begin{verbatim}
# python3 coverage_checking_table_result.py
\end{verbatim}
%
The command prints out a table of analysis time:
\begin{verbatim}
Name                       Path                                   LOC  Check Time
-------------------------  -----------------------------------  -----  ------------
branching                  branching/branching-covered             46  695.652us
...
recur-covered-misaligned   recur/recur-covered-misaligned          83  8.843ms
\end{verbatim}
%
The analysis time reported in this table is the sum of the covered and
uncovered versions.

The command also writes out the coverage-checking results to a log file at
\begin{verbatim}
GuideTypes/bench/coverage-check/table-coverage-checking.log
\end{verbatim}
%
To view the first twenty-four lines of the log file, where we have the results
for benchmark \texttt{branching}, run
\begin{verbatim}
# head -24 table-coverage-checking.log
\end{verbatim}
%
Under the header \texttt{Benchmark branching covered} (i.e., the covered version
of the benchmark \texttt{branching}), the result of coverage checking is
\begin{verbatim}
The final type is fully covered
\end{verbatim}
as expected.
%
Hence, the result is \textcolor{blue}{True Positive}, as reported on the first
line of the seventh column of Table~1.
%
On the other hand, under the header \texttt{Benchmark branching uncovered}
(i.e., the uncovered version of the benchmark \texttt{branching}), the result of
coverage checking is
\begin{verbatim}
The final type is not fully covered
\end{verbatim}
as expected.

The coverage-checking algorithm raises exceptions for the misaligned versions of
the context-free (i.e., last four) benchmarks.
%
The exceptions indicate that the coverage-checking algorithm has detected
misalignment of guide coroutines with respect to procedure calls.
%
The exceptions are printed on the Docker container's terminal.

As a result of the exceptions, the coverage-checking results in the log file are
empty.
%
For example, take a look at the last eighteen lines of the log file by running
\begin{verbatim}
# tail -18 table-coverage-checking.log
\end{verbatim}
%
Under both the headers \texttt{Benchmark recur-misaligned covered} (i.e., the
misaligned and covered version of the benchmark \texttt{recur}) and
\texttt{Benchmark recur-misaligned uncovered}, the coverage-checking results are
empty.

We interpret these exceptions as the result Uncovered/Negative, but this is only
correct for those cases where the ground truth is also Uncovered.
%
Conversely, if the ground truth is Covered, the coverage-checking result is
\textcolor{red}{False Negative} as highlighted in red in the seventh column of
Table~1.
