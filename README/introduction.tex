% !TeX root = README.tex

\section{Introduction}

\paragraph{What is included}

From Zenodo, you should obtain the following two items:
\begin{itemize}
  \item This \texttt{README.pdf} document
  \item Archived and compressed Docker image \texttt{guide\_types.tar.gz}.
\end{itemize}

\paragraph{Artifact overview}

Our artifact is a program analysis tool with three functionalities:
\begin{enumerate}[label=(F\arabic*), ref=F\arabic*]
  \item Type-equality checking: check structural type equality of guide types
        \label{functionality:type-equality checking}
  \item Type inference: infer guide types of model and guide coroutines (using
        the first functionality for structural-type-equality checking)
        \label{functionality:type inference}
  \item Coverage checking: check whether the support of sequentially composed
        guide coroutines coincides with the support of a model coroutine.
        \label{functionality: coverage checking}
\end{enumerate}

In probabilistic programming with our new coroutine-based programmable inference
framework, the user provides (i) a model coroutine and (ii) a sequential
composition of guide coroutines.
%
The model coroutine specifies a probabilistic model for Bayesian inference.
%
Meanwhile, the sequential composition of guide coroutines customizes the Block
Metropolis-Hastings (BMH) algorithm, where we successively run the guide
coroutines, each of which is followed by an MH acceptance routine.
%
Each guide coroutine only updates a subset (i.e., block) of random variables.
%
The model and guide coroutines communicate with one another by message passing,
and their communication protocols are described by guide types.
%
To algorithmically decide structural type equality of guide types, our artifact
implements the bisimilarity-checking algorithm by \citeauthor{Hirshfeld1994} for
context-free processes with finite norms (\ref{functionality:type-equality
  checking}).
%
This structural-type-equality checking algorithm is also incorporated into the
guide-type-inference algorithm (\ref{functionality:type inference}).
%
Finally, to verify the soundness of the BMH algorithm, our artifact checks
whether the support of the sequentially composed guide coroutines covers all
possible traces admitted by the model coroutine (\ref{functionality: coverage
  checking}).

\paragraph{Supported claims in the paper}

The artifact is used to produce Table~1 in the paper:
\begin{itemize}
  \item Analysis time of guide-type inference (i.e., the fifth column in the
        table)
  \item Results and analysis time of coverage checking (i.e., the last three
        columns in the table).
\end{itemize}
